Time series data arise in different areas of science and technology, describing
the \emph{behavior} of both natural and man-made systems over time. These
behaviors are often quite complex with uncertainty, which in turn require us to
incorporate sophisticated dynamics and stochastic models to model such
behavioral patterns. Furthermore, these complex behaviors can
\emph{change} over time due to some external event and/or some internal
systematic change of dynamics/distribution. For example, consider the case of
monitoring one's physical activity via an array of accelerometer body sensors
over time. A certain pattern emerges on the time series of the sensors' readings
while the person is walking; however, this pattern quickly changes to a new one
as the person starts running. From the data anaylsis perspective, firstly it is
important to detect these \emph{change points} as they are quite often
indicative of an ``interesting'' event or an anomaly in the system. Secondly, we
are also interested to characterize the new \emph{state} of the system (e.g. running vs.
walking) which reflects its modus operandi. Change point detection methods ~\cite{} have been proposed
to answer the first question while the classical Hidden Markov Models (HMM) can
answer both.

One crucial observation in many real-world systems (natural and man-made) is
that the behavior changes are typically infrequent; that is, the system takes some
(unknown) time before it changes its behavior to a new modus operandi. For
instance, in our earlier example, it is unlikely that a person changes between
walking and running very frequently, plus the durations of different activities
over time are relatively long and highly variable. We refer to this as the
\emph{inertial property}, alluding to the physical property of matter that ensures it will continue along
a fixed course unless acted upon by an external force. Unfortunately,
the classical HMMs are not equipped with sufficient mechanisms to capture this
property and quite often result in a high rate of false positives in terms of
change points. 
