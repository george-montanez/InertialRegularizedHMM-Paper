Time series data arise in different areas of science and technology, describing
the \emph{behavior} of both natural and man-made systems over time. These
behaviors are often quite complex with uncertainty, which in turn require us to
incorporate sophisticated dynamics and stochastic models to model them.
Furthermore, these complex behaviors can \emph{change} over time due to some external event and/or some internal
systematic change of dynamics/distribution. For example, consider the case of
monitoring one's physical activity via an array of accelerometer body sensors
over time. A certain pattern emerges on the time series of the sensors' readings
while the person is walking; however, this pattern quickly changes to a new one
as the person starts running. From the data anaylsis perspective, firstly it is
important to detect these \emph{change points} as they are quite often
indicative of an ``interesting'' event or an anomaly in the system. Secondly, we
are also interested to characterize the new \emph{state} of the system (e.g. running vs.
walking) which reflects its modus operandi. Change point detection methods ~\cite{} have been proposed
to answer the first question while the classical Hidden Markov Models (HMM) can
answer both.

One crucial observation in many real-world systems (natural and man-made),
however, is that the behavior changes are typically infrequent; that is, the
system takes some (unknown) time before it changes its behavior to a new modus operandi. For
instance, in our earlier example, it is unlikely that a person changes between
walking and running very frequently, making the durations of different
activities over time relatively long and highly variable. We refer to this as the
\emph{inertial property}, alluding to the physical property of matter that
ensures it will continue along a fixed course unless acted upon by an external
force. Unfortunately, the classical HMMs are not equipped with sufficient
mechanisms to capture this property and quite often result in a high rate of
state transitioning and subsequently false positives in terms of detecting
change points.

There are very few solutions in the literature to address this problem. In
the context of Markov models, Fox \emph{et al.}~\cite{fox2011sticky} have
recently proposed the \emph{sticky hierarchical Dirichlet process hidden Markov
model (HDP-HMM)} which uses a Bayesian non-parametric approach with appropriate
priors to promote self-transitioning (or \emph{stickiness}) for HMMs. Albeit its
neat theoretical foundations, HDP-HMM is not a practical solution in many real-world situations.
In particular, the performance of HDP-HMM tends to break down as the
dimentionality of the problem goes beyond 10. Moreover, due to iterative Gibbs
sampling for its learning, HDP-HMM is computationally prohibitive. But, the
most significant downside of HDP-HMM in practice originates from its
non-parametric Bayesian nature: due to the existence of many hyperparameters,
the search space for initial tuning is exponentially large which significantly
affects the learning quality for a given task.

In this paper, we propose a regularization-based framework for HMMs called
\emph{Inertial HMM} to bias them toward the inertial property. Similar to
HDP-HMM, our framework is based on theoretically sound foundations, yet much
simpler and more intuitive than HDP-HMM. In particular, our framework has only
two initial parameters for which we have developed intuitive initilization
techniques that significantly minimizes the effort needed for parameter tuning.
Furthermore, as we show later, in practice, our proposed
methodology boils down to upgraded update rules for standard HMMs. The main
practical implication of this observation is that using our methodology, the
standard HMM packages can be simply upgraded to support the inertial property yet preserve the
computational efficiency of the standard HMM approach. By performing rigouros
experiments on both synthetic and moderate dimensional real datasets, we show
that not only are Inertial HMMs much faster than HDP-HMM, but the quality of
detection is siginificantly better than that of the HDP-HMM, and therefore
proposing that Inertial HMMs are far more practical choices compared to the
state-of-the-art..
